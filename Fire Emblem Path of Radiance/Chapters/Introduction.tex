\section*{Introduction}

Original notes can be found in this link: \url{https://goo.gl/kw4XWT}
\newline\newline
\textbf{\large{General Notes and Tips}}
\newline\newline
\begin{itemize}
    \item There’s no RNG manipulation in this run. As such, you can feel free to do whatever cursor movement you like.
    \item All X/Y switching strategies in this route are up to personal preference, though generally speaking you want to at least follow the order of characters used each turn.
    \item These notes assume fixed growths mode where the stats of characters are always the same, assuming you follow the route. The route will not be consistent nor hold if you decide to run random mode. More details about fixed growths can be found \href{https://serenesforest.net/path-of-radiance/general/fixed-mode/}{here}.
    \item Don’t skip cutscenes for opening chests, this is slower than letting them play out.
    \item Do skip cutscenes for opening doors, though.
    \item Cursor Movement - You should be holding B anytime you select and move the cursor for longer distances. Hold B before selecting a character so you don’t accidentally cancel.
    \begin{itemize}
        \item D-pad is useful for straight rigid movement. The cursor will always be stopped by terrain / enemy units and the current character’s movement range.
        \item Analog stick is useful for diagonal movement and/or moving through terrain, since the analog stick will never be stopped by terrain / enemy units. However, it also isn’t stopped by the character’s movement range.
    \end{itemize}
    \item You should be doing the FEP (Fast Enemy Phase) glitch on every turn in the run (this makes the camera move more quickly between non-player units, saving minutes overall).
    \begin{itemize}
        \item Holding B as your last action ends, then opening the menu and ending turn while holding B should always work.
        \item If you did a character movement without holding B, it’s better to open the main menu while holding B and a direction (like FEP in radiant dawn).
    \end{itemize}
    \item On Enemy Phase, holding down the A button will make it so the white box around a character will not appear.
    \item Mashing A and Start clears the exp bar faster (both in a level and bonus exp in base).
    \item Mashing start clears the level up screen faster.
    \item X = do an X-switch.
    \begin{itemize}
        \item Pressing X on a character will jump the cursor to the next unused character in the unit list.
        \item Pressing X on an empty tile will jump the cursor to the top unused character in the unit list.
        \item You should be holding B during all X switches so the camera move quickly.
        \item Note abbreviations for X-switching:
        \begin{itemize}
            \item “2X” = press X twice to jump to Character
            \item “Off X” = press X on an empty tile to jump to Character
            \item “X on Character1” = move cursor to Character1, press X to jump to Character2
        \end{itemize}
    \end{itemize}
    \item Y = do a Y-switch.
    \begin{itemize}
        \item Pressing Y on a character brings up the character screen. This is primarily useful for going backwards in the unit list since it skips moving the camera when you cancel. For example, if the top of the Unit List is Ike, Marcia, Tanith, and you want to select Marcia after moving Tanith, you press Y on Tanith, press up once, then press B to instantly select Marcia.
    \end{itemize}
\end{itemize}